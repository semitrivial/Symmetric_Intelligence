\documentclass{article}
\usepackage[utf8]{inputenc}

\title{Negative Intelligence: An Improvement to Legg-Hutter Universal Intelligence}
\author{Samuel Allen Alexander}

\begin{document}

\maketitle

\begin{abstract}
    Fill this in.
\end{abstract}

\section{Introduction}

In their well-known paper \cite{legg}, Legg and Hutter write:
\begin{quote}
    ``As our goal is to produce a definition of intelligence that is as broad and
    encompassing as possible, the space of environments used in our definition should
    be as large as possible.''
\end{quote}
We find it strange, then, that they choose to limit their environments to have
rewards in $[0,1]\cap \mathbb Q$, the set of non-negative rational numbers
$\leq 1$. The decision to limit rewards to a closed interval does make a lot of sense
in the context of their paper, but the decision to rule out negative rewards does
not make as much sense.

We propose that for the purpose of measuring intelligence,
one should consider agents' performance in environments that punish agents as well
as reward them. We have discovered that when negative rewards are allowed, this
introduces a certain algebraic structure into the agent-environment framework, which
in turn allows an intelligence definition that is both better-behaved and also
easier to compute. The main objection we anticipate to our proposed improved
intelligence measure is that it assigns negative intelligence to certain agents.
We will argue that this not only makes perfect sense, in fact when one looks at
it from the proper perspective, it is almost nonsensical to demand otherwise.

\end{document}
