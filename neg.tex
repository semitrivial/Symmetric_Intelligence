\documentclass{article}
\usepackage[utf8]{inputenc}

\title{Negative Intelligence: An Improvement to Legg-Hutter Universal Intelligence}
\author{Samuel Allen Alexander}

\begin{document}

\maketitle

\begin{abstract}
    Fill this in.
\end{abstract}

\section{Introduction}

In their well-known paper \cite{legg}, Legg and Hutter write:
\begin{quote}
    ``As our goal is to produce a definition of intelligence that is as broad and
    encompassing as possible, the space of environments used in our definition should
    be as large as possible.''
\end{quote}
So motivated, we investigated what would happen if we extended the universe
of environments to include environments with rewards from $\mathbb Q\cap [-1,1]$
instead of just from $\mathbb Q\cap [0,1]$ as in Legg and Hutter's paper.
In other words, we investigated what would happen if environments are not only
allowed to reward agents but also to punish agents (a punishment being a negative
reward).

We discovered that when negative rewards are allowed, this
introduces a certain algebraic structure into the agent-environment framework, which
in turn allows an intelligence definition that is both better-behaved and also
easier to compute. The main objection we anticipate to our proposed improved
intelligence measure is that it assigns negative intelligence to certain agents.
We will argue that this not only makes perfect sense, in fact when one looks at
it from the proper perspective, it is almost nonsensical to demand otherwise.



\end{document}
